\documentclass{llncs}

\sloppy

% Packages LaTeX
% Packages.tex

% Packages di latex utilizzati 

  % Standard dell'ams
\usepackage{amssymb,amsmath}

  % Per le definizioni di connessioni di Galois
\usepackage{galois}

  % Per formattare ed includere il testo
\usepackage{lgrind}
     % Cambiamento dei font per lgrind: tutti in Courier
\def\CMfont{\ttfamily\itshape}
\def\KWfont{\ttfamily\bfseries}
\def\VRfont{\ttfamily}
\def\BGfont{\ttfamily}
\def\NOfont{\ttfamily}

  % Caratteri speciali per i nomi di funzione
\usepackage{bbm}

  % Caratteri ``cal'' carini
\usepackage[mathcal]{euscript}

  % per \baro
\usepackage{stmaryrd}

  % per \xspace
\usepackage{xspace}

  % per i diagrammi
\usepackage[all]{xy}

  % for scalebox
\usepackage{graphicx}
 
   % Per le figure incastrate l'una nell'altra
\usepackage{subfigure}

  % Per le figure nel testo
\usepackage{wrapfig}


\usepackage{tabularx}

  % Per includere i verbatim
\newbox\subfigbox
\makeatletter
\newenvironment{subfloat}
{\def\caption##1{\gdef\subcapsave{\relax##1}}%
\let\subcapsave\@empty
\setbox\subfigbox\hbox
\bgroup}
{\egroup
\subfigure[\subcapsave]{\box\subfigbox}}
\makeatother

   % Per la compilazione Condizionale
\usepackage{ifthen}

\usepackage{empheq}

   % Per i riferimenti incrociati



%\ifx\pdfoutput\undefined % Se compilo con ``latex'' ed uso ps2pdf
%\usepackage{color}
%\usepackage[dvips]{graphicx}       %%% graphics for dvips
%\DeclareGraphicsExtensions{.eps}   %%% standard extension for included graphics
%\usepackage[bookmarks=true]{hyperref}
%\else                    % Se compilo con ``pdflatex'' 
%\usepackage[pdftex]{graphicx}      %%% graphics for pdfLaTeX 
%\DeclareGraphicsExtensions{.pdf}   %%% standard extension for included graphics
%\usepackage[pdftex]{thumbpdf}      %%% thumbnails for pdflatex
%\usepackage[pdftex,                %%% hyper-references for pdflatex
%bookmarks=true,%                   %%% generate bookmarks ...
%bookmarksnumbered=true,%           %%% ... with numbers
%hypertexnames=false,%              %%% needed for correct links to figures !!!
%breaklinks=true,%                  %%% break links if exceeding a single line
%linkbordercolor={0 0 1}]{hyperref} %%% blue frames around links
%                                  %%% pdfborder={0 0 1} is the default
%\hypersetup{
%pdfauthor   = {Francesco Logozzo \& Agostino Cortesi},
%pdftitle    = {to be decided},
%pdfsubject  = {Objects},
%pdfkeywords = {Abstract Interpretation, Static Analysis, Object Oriented}
%}
%\pdfadjustspacing=1                %%% force LaTeX-like character spacing
%\fi 


%%% Local Variables: 
%%% mode: plain-tex
%%% TeX-master: "main"
%%% End: 


% Definizioni di Macro ed ambienti
% Definizione dei simboli

\newcommand{\eg}{\textit{e.g.}}
\newcommand{\ie}{\textit{i.e.}}

% Definizione dell' operatore di astrazione
\newcommand{\abs}[1]{\ensuremath{\bar{\mathsf{#1}}}}

\newcommand{\sx}{\llbracket}
\newcommand{\dx}{\rrbracket}

  % Semantica generica
\newcommand{\sem}[1]{\ensuremath{\sx \mathtt{#1} \dx}}

  % Semantica che prende un nome
\newcommand{\semantica}[2]{\ensuremath{\mathbb{#1}\sem{#2}}}

\newcommand{\BigSemantica}[2]{\ensuremath{\mathbb{#1}{\Bigg\llbracket #2 \Bigg\rrbracket}}}

  % Semantiche generiche
\newcommand{\csem}[1]{\ensuremath{\semantica{s}{#1}}}
\newcommand{\asem}[1]{\ensuremath{\semantica{\abs{s}}{#1}}}
\newcommand{\asemRefined}[1]{\ensuremath{\semantica{\abs{s}^*}{#1}}}

  % Funzione di ``transfer''
\newcommand{\trasf}[1]{\ensuremath{\semantica{t}{#1}}}
\newcommand{\atrasf}[1]{\ensuremath{\semantica{\abs{t}}{#1}}}

  % Semantica di: costruttore,  metodo
\newcommand{\semcostr}[1]{\semantica{i}{#1}}
\newcommand{\semmetodo}[1]{\semantica{m}{#1}}
\newcommand{\semoggetto}[1]{\semantica{o}{#1}}
\newcommand{\semclasse}[1]{\semantica{c}{#1}}

  % Semantica Colleting Traces
\newcommand{\Semmetodo}[1]{\semantica{M}{#1}}
\newcommand{\Semoggetto}[1]{\semantica{O}{#1}}
\newcommand{\Semclasse}[1]{\semantica{C}{#1}}

  % Semantics Reachable States
\newcommand{\rsemcostr}[1]{\semantica{I}{#1}}
\newcommand{\rsemmetodo}[1]{\Semmetodo{#1}}
\newcommand{\rsemoggetto}[1]{\semantica{O}{#1}}
\newcommand{\rsemclasse}[1]{\semantica{C}{#1}}

  % Seamntica Astratta
\newcommand{\asemcostr}[1]{\semantica{\bar{I}}{#1}}
\newcommand{\asemmetodo}[1]{\semantica{\bar{M}}{#1}}
\newcommand{\asemoggetto}[1]{\semantica{\bar{O}}{#1}}
\newcommand{\asemclasse}[1]{\semantica{\bar{C}}{#1}}

  % Operazioni su stati
\newcommand{\less}{\ensuremath{\sqsubseteq}}
\newcommand{\join}{\ensuremath{\sqcup}}
\newcommand{\meet}{\ensuremath{\sqcap}}
\newcommand{\bottom}{\ensuremath{\bot}}
\newcommand{\bigjoin}{\ensuremath{\bigsqcup}}

  % Semantica astratta
\newcommand{\asemantica}[2]{\semantica{\bar{#1}}{#2}}
\newcommand{\asemanticaBig}[2]{\BigSemantica{\bar{#1}}{#2}}


  % Semantica astratta BIS, per aggiungere pedici
\newcommand{\asemanticaBis}[3]{\semantica{\bar{#1}_{\textit{#2}}}{#3}}

  % Operazioni astratte
\newcommand{\nabs}[1]{\bar{#1}}
\newcommand{\abot}{\nabs{\bot}}
\newcommand{\atp}{\nabs{\ensuremath{\top}}}
\newcommand{\aless}{\nabs{\less}}
\newcommand{\asup}{\ensuremath{\nabs{\sqsupseteq}}}
\newcommand{\acup}{\ensuremath{\nabs{\sqcup}}}
\newcommand{\ajoin}{\acup}
\newcommand{\abigcup}[1]{\nabs{\bigsqcup_{#1}}}
\newcommand{\abigjoin}[1]{\abigcup{#1}}
\newcommand{\acap}{\ensuremath{\nabs{\sqcap}}}
\newcommand{\ameet}{\acap}
\newcommand{\abigcap}[1]{\nabs{\bigsqcap_{#1}}}
\newcommand{\abigmeet}[1]{\abigcap{#1}}


 % lfp : least fixpoint
\newcommand{\lfp}[2]{\ensuremath{\mathrm{lfp}^{#1}_{#2}}}

 % gfp: greatest fixpoint
\newcommand{\gfp}[2]{\ensuremath{\mathrm{gfp}^{#1}_{#2}}}


% Insiemi e Funzioni

  % Insieme, elemento di insieme, nessun elemento
\newcommand{\insieme}[1]{\ensuremath{\mathsf{#1}}}
\newcommand{\el}[1]{\ensuremath{\mathsf{#1}}}
\newcommand{\noel}{\el{\bot}}

  % Insieme delle parti
\newcommand{\parti}[1]{\ensuremath{\mathcal{P}{(#1)}}}
\newcommand{\partiFin}[1]{\ensuremath{\mathcal{P}_{\mathrm{fin}}{(#1)}}}


  % Elementi di insiemi
\newcommand{\noval}{\ensuremath{\baro}}

  % Cardinalita di inisiemi
\newcommand{\cardinalita}[1]{\ensuremath{\# #1}}

  % Env, Store, Addr, Val, Label : Scorciatoie
\newcommand{\env}{\insieme{Env}}
\newcommand{\Env}{\env}

\newcommand{\aenv}{\ensuremath{\bar{\env}}}

\newcommand{\store}{\insieme{Store}}
\newcommand{\Store}{\store}

\newcommand{\astore}{\ensuremath{\overline{\store}}}

\newcommand{\AState}{\abs{\Sigma}}
\newcommand{\astate}{\AState}

\newcommand{\addr}{\insieme{Addr}}
\newcommand{\Addr}{\addr}

\newcommand{\aaddr}{\ensuremath{\overline{\addr}}}

\newcommand{\Reference}{\insieme{Ref}}
\newcommand{\reference}{\reference}

\newcommand{\AReference}{\ensuremath{\overline{\Reference}}}

\newcommand{\val}{\insieme{Val}}

\newcommand{\etichetta}{\insieme{Label}}
\newcommand{\cont}{\ensuremath{\kappa}}

  % Stati
\newcommand{\stati}[1]{\ensuremath{\Sigma}}
  % Tracce
\newcommand{\tracce}[1]{\ensuremath{\mathcal{T}(#1)}}
  % Transizione
\newcommand{\tr}[1]{\ensuremath{\xrightarrow{#1}}}

  % Insieme di funzioni
\newcommand{\funzione}[2]{\ensuremath{[#1 \rightarrow#2]}}

  % Funzione vuota
\newcommand{\emptyfun}{\ensuremath{[\cdot]}}

  % Insieme di funzioni monotone e joim-morphism
\newcommand{\funmon}[2]{\ensuremath{[#1 \stackrel{m}{\longrightarrow} #2]}}
\newcommand{\funjoin}[2]{\ensuremath{[#1 \stackrel{\sqcap}{\longrightarrow} #2]}}

% Astratto
  % Dominio concreto
\newcommand{\dom}[1]{\insieme{#1}}

  % Dominio astratto
\newcommand{\adom}[1]{\ensuremath{{\mathsf{#1}}}}

 % Vari domini
\newcommand{\Intervals}{\dom{Intv}}
\newcommand{\LT}{\dom{LT}}
\newcommand{\Boxes}{\dom{Boxes}}
\newcommand{\DBM}{\dom{DBM}}
\newcommand{\Octagons}{\dom{Oct}}
\newcommand{\Polyhedra}{\dom{Poly}}
\newcommand{\Poly}{\Polyhedra}
\newcommand{\Lineq}{\dom{LinEq}}
\newcommand{\LinEq}{\Lineq}
\newcommand{\Karr}{\Lineq}
\newcommand{\Symbolic}{\dom{Symb}}
\newcommand{\Pentagons}{\dom{Pnt}}
\newcommand{\SubPolyhedra}{\dom{SubPoly}}
\newcommand{\SubPoly}{\SubPolyhedra}
\newcommand{\Subpoly}{\SubPolyhedra}

  % Elemento di dominio astratto
\newcommand{\ael}[1]{\ensuremath{\abs{\el{#1}}}}
\newcommand{\aelBis}[2]{\ensuremath{\abs{\el{#1}}_\el{#2}}}

  % Funzione di astrazione e concretizzazione
\newcommand{\alfa}{\ensuremath{\alpha}}
%\renewcommand{\gamma}{\ensuremath{\gamma}}
\newcommand{\gm}{\ensuremath{\gamma}}


% Funzioni usate nella tesi
  % Transizione basica
\newcommand{\prossimo}{\ensuremath{\mathrm{next}}}
%\newcommand{\next}{\ensuremath{\mathrm{next}}}
\newcommand{\nextDir}{\ensuremath{\mathrm{next_{dir}}}}
\newcommand{\nextInd}{\ensuremath{\mathrm{next_{ind}}}}
\newcommand{\prossimoDir}{\nextDir}
\newcommand{\prossimoInd}{\nextInd}

  % Transizione ``collecting''
\newcommand{\Next}{\ensuremath{\mathrm{Next}}}
\newcommand{\NextDir}{\ensuremath{\mathrm{Next_{dir}}}}
\newcommand{\NextInd}{\ensuremath{\mathrm{Next_{ind}}}}

  % Reachable addresses
\newcommand{\reachable}{\ensuremath{\mathrm{reachable}}}
  % Update function
\newcommand{\update}{\ensuremath{\mathrm{update}}}
  % Storia delle interazioni
%\newcommand{\etichette}{\ensuremath{\mathrm{labels}}}
\newcommand{\storia}{\ensuremath{\mathrm{history}}}

  % Contesto
\newcommand{\contesto}{\ensuremath{\mathrm{Context}}}

  % Contesto Astratto
\newcommand{\acontesto}{\ensuremath{\mathrm{\overline{Context}}}}

% Tracce : definizioni
\newcommand{\stringavuota}{\ensuremath{\epsilon}}


% Funzioni di astrazione
  % Prima Astrazione, quella dei metodi
\newcommand{\alfaPrima}{\ensuremath{\alpha_{\Yright}}}
\newcommand{\gammaPrima}{\ensuremath{\gamma_{\Yright}}}
  % Versione higher-order
\newcommand{\alfaPrimaDot}{\ensuremath{\dot{\alpha}_{\Yright}}}

  % Seconda Astrazione, quella degli stati
\newcommand{\alfaSeconda}{\ensuremath{\alpha_{\circ}}}
\newcommand{\gammaSeconda}{\ensuremath{\gamma_{\circ}}}
  % Versione higher-order
\newcommand{\alfaSecondaDot}{\ensuremath{\dot{\alpha}_{\circ}}}


\newcommand{\alfaAddr}{\ensuremath{\alpha_\el{a}}}
\newcommand{\gammaAddr}{\ensuremath{\gamma_\el{a}}}

\newcommand{\gammaBool}{\ensuremath{\gamma_\el{b}}}

\newcommand{\gammaEnv}{\ensuremath{\gamma_\el{e}}}

\newcommand{\gammaOct}{\ensuremath{\gamma_\el{o}}}
\newcommand{\gammaDOct}{\ensuremath{\gamma_\el{do}}}

\newcommand{\gammaRef}{\ensuremath{\gamma_\el{r}}}

\newcommand{\gammaStore}{\ensuremath{\gamma_\el{s}}}

\newcommand{\gammaState}{\ensuremath{\gamma_{\astate}}}

% Estensioni per OO

  % Classe
\newcommand{\classe}[1]{\ensuremath{\mathtt{#1}}}
\newcommand{\classi}{\ensuremath{\mathcal{C}}}

  % Gerarchia
\newcommand{\gerarchia}[1]{\ensuremath{\mathcal{#1}}}
\newcommand{\gerarchie}{\ensuremath{\mathbb{H}}}

  % Operatori su Gerarchie 
\newcommand{\inserisciClasse}{\ensuremath{\uplus}}
\newcommand{\costruisciClasse}{\ensuremath{\beta}}
\newcommand{\unisciGerarchie}{\ensuremath{\Cup}}

\newcommand{\aclasse}[1]{\ensuremath{\abs{\mathtt{#1}}}}
\newcommand{\aclasseLong}[1]{\ensuremath{\overline{\mathtt{#1}}}}
  % Oggetto, instanza di classe
\newcommand{\oggetto}[1]{\ensuremath{\mathsf{#1}}}
  % Metodo
\newcommand{\metodo}[1]{\ensuremath{\mathtt{#1}}}
  % Campo
\newcommand{\campo}[1]{\ensuremath{\mathtt{#1}}}

  % Astrazioni di campi
\newcommand{\acampo}[1]{\ensuremath{\mathtt{\abs{#1}}}}

  % Astrazione di metodi con constraints
\newcommand{\ametodo}[1]{\ensuremath{\mathtt{\abs{#1}}}}

% Dominio delle Tracce Collecting

\newcommand{\adomTracce}{\ensuremath{\adom{D}_{\Yright}}}

\newcommand{\atopTracce}{\ensuremath{\abs{\top}_{\Yright}}}
\newcommand{\abottomTracce}{\ensuremath{\abs{\bot}_{\Yright}}}

\newcommand{\alessTracce}{\ensuremath{\abs{\subseteq}_{\Yright}}}
\newcommand{\alessTraccia}{\ensuremath{\abs{\subseteq}^{\tau}_{\Yright}}}

\newcommand{\ajoinTracce}{\ensuremath{\abs{\cup}_{\Yright}}}
\newcommand{\ajoinTraccia}{\ensuremath{\abs{\cup}^{\tau}_{\Yright}}}

\newcommand{\ameetTracce}{\ensuremath{\abs{\cap}_{\Yright}}}
\newcommand{\ameetTraccia}{\ensuremath{\abs{\cap}^{\tau}_{\Yright}}}

% Scorciatoie

  % ``Prende''
\newcommand{\prende}{\ensuremath{\mapsto}}

  % Simbolo di definizione
\newcommand{\df}{\ensuremath{\triangleq}}

  % Tupla
\newcommand{\tupla}[1]{\ensuremath{\langle #1 \rangle}}

  % Stati Iniziali
\newcommand{\statiiniziali}[1]{\ensuremath{S_0\tupla{#1}}}

  % Modularita
\newcommand{\hasA}{has-A}
\newcommand{\isA}{is-A}

  % Riferimento ad una formula
\newcommand{\formula}[1]{\ensuremath{\mathrm{(\ref{#1})}}}

% Complessita

  % O-notation
\newcommand{\costo}[1]{\ensuremath{\kappa_{#1}}}


% Definizioni per i domini di vincoli

\newcommand{\Vars}{\ensuremath{\mathtt{Vars}}}
\newcommand{\C}{\ensuremath{\dom{C}}}
\newcommand{\initC}{\ensuremath{\mathrm{initial_{\C}}}}

\newcommand{\dominio}{\ensuremath{\mathrm{dom}}}
\newcommand{\range}{\ensuremath{\mathrm{range}}}

  % Relazioni
\newcommand{\rel}[1]{\ensuremath{\mathsf{\rho}[\mathtt{#1}]}}
\newcommand{\conDuepar}[2]{\ensuremath{\con'[\mathtt{#1}, \mathtt{#2}]}}
\newcommand{\conTrepar}[3]{\ensuremath{\con[\mathtt{#1}, \allowbreak \mathtt{#2}, \allowbreak \mathtt{#3}]}}
\newcommand{\relass}[1]{\ensuremath{\rho(#1)}}

\newcommand{\rgamma}{\ensuremath{\gamma_{\rho}}}
\newcommand{\cgamma}{\ensuremath{\gamma_{\C}}}

 % Lettera
\newcommand{\con}{\ensuremath{\mathsf{c}}}

 % Sequenza
\newcommand{\seq}[1]{\ensuremath{\vartriangleright_\mathtt{#1}}}


 % Semantiche per i vincoli
\newcommand{\semc}[1]{\semantica{s}{#1}}

\newcommand{\semtracce}[1]{\semantica{t}{#1}}
\newcommand{\asemtracce}[1]{\semantica{\bar{t}}{#1}}
\newcommand{\asemtracces}[1]{\semantica{\bar{s}}{#1}}


\newcommand{\modsem}[1]{\ensuremath{\semantica{T}{#1}}}
\newcommand{\amodsem}[1]{\semantica{\bar{T}}{#1}}

\newcommand{\A}{\ensuremath{\adom{A}}\xspace}
\newcommand{\modA}{\ensuremath{\adom{A}_\mathsf{M}}\xspace}
\newcommand{\Absel}{\ensuremath{\ael{a}}\xspace}
\newcommand{\modAbsel}{\ensuremath{\ael{a}_\mathsf{m}}\xspace}

\newcommand{\relsem}[1]{\semantica{R}{#1}}

 % Proiezione/dropping

\newcommand{\lascia}{\ensuremath{\delta}}
\newcommand{\tieni}{\ensuremath{\pi}}

%\newcommand{\Absless}{\ensuremath{\sqsubseteq^\mathsf{a}}}
\newcommand{\modAbsless}{\ensuremath{\abs{\sqsubseteq}_\mathsf{m}}}

%\newcommand{\Absjoin}{\ensuremath{\sqcap^\mathsf{a}}}
\newcommand{\modAbsjoin}{\ensuremath{\abs{\sqcap}_\mathsf{m}}}

\newcommand{\aconc}{\ensuremath{\abs{\odot}}\xspace}


 % Operazioni sul dominio di vincoli
\newcommand{\cless}{\ensuremath{\preceq}}
\newcommand{\cgreater}{\ensuremath{\succeq}}
\newcommand{\ctop}{\ensuremath{\vernal}}
\newcommand{\cbot}{\ensuremath{\bot^\con}}
\newcommand{\cmeet}{\ensuremath{\curlywedge}}
\newcommand{\cjoin}{\ensuremath{\curlyvee}}
\newcommand{\cwiden}{\ensuremath{\hbox{\hbox to 0pt{\raisebox{4pt}{$\relbar$}}$\curlyvee$}}}
\newcommand{\bigcjoin}{\ensuremath{\bigcurlyvee}}

\newcommand{\alg}{\ensuremath{\mathcal{A}}}

 % Definizione di BodyOf
\newcommand{\bodyof}[1]{\ensuremath{\mathrm{bodyOf}(\mathbf{#1})}}

 % Definitione di tiedVars
\newcommand{\tiedvars}{\ensuremath{\mathrm{tiedVars}}}

  % Operazioni su Tracce
%\newcommand{\tracce}[1]{\ensuremath{\wp(#1^\infty)}\xspace}
\newcommand{\iniziatracce}[2]{\ensuremath{\mathcal{T}(#1,#2)}}
\newcommand{\giunzione}{\ensuremath{\frown}\xspace}
\newcommand{\conc}{\ensuremath{\odot}\xspace}
\newcommand{\last}{\ensuremath{\mathrm{last}}}

  % Funzioni Semantiche ``Astratte''
\newcommand{\flow}[1]{\ensuremath{\mathrm{flow}_\mathtt{#1}}}
\newcommand{\aflow}[1]{\ensuremath{\overline{\mathrm{flow}}_\mathtt{#1}}}

  % Definizione di ``approssima'', i.e. un vincolo approssima la semantica di un metodo
\newcommand{\approssima}{\ensuremath{\Subset}}

  % Approssimazione con vincoli della semantica di un metodo
\newcommand{\conmetodo}[5]{\ensuremath{\con_{#1}[\mathtt{#2}, \allowbreak \mathtt{#3}, \allowbreak \mathtt{#4}, \allowbreak \mathtt{#5}]}}



% Definizione del dominio degli osservabili
\newcommand{\osservabile}[2]{\ensuremath{\mathcal{O}_{#2}(\classe{#1})}}
\newcommand{\oless}{\ensuremath{{\sqsubseteq}_{o}}}
\newcommand{\osup}{\ensuremath{{\sqsupseteq}_{o}}}
\newcommand{\ojoin}{\ensuremath{{\sqcup}_{o}}}
\newcommand{\omeet}{\ensuremath{{\sqcap}_{o}}}
\newcommand{\otp}{\ensuremath{{\top}_{o}}}
\newcommand{\obot}{\ensuremath{{\bot}_{o}}}
\newcommand{\obigjoin}{{{\bigsqcup}{}}_{o}}

\newcommand{\domPiuPreciso}{\ensuremath{O[\parti{\Sigma}]}}

% Interfaccia di una classe
\newcommand{\interfaccia}[1]{\ensuremath{\iota(\classe{#1})}}


% Definizione dell' insieme delle astrazioni di un dominio +
% operazioni

\newcommand{\astrazioni}[1]{\ensuremath{\mathcal{A}(#1)}}
\newcommand{\dless}{\ensuremath{\leq}}
\newcommand{\djoin}{\ensuremath{\vee}}
\newcommand{\dmeet}{\ensuremath{\wedge}}

% Dominio dei valori
\newcommand{\dominioVal}[1]{\ensuremath{\mathit{#1}}}

% Stato interno ad un oggetto
\newcommand{\stato}{\ensuremath{\sigma}}

% Semantica backward concreta ed astratta
\newcommand{\backSemmetodo}[1]{\semantica{{M}^{<}}{#1}}
\newcommand{\abacksemmetodo}[1]{\semantica{\bar{M}^{<}}{#1}}


% Complexity
\newcommand{\Order}[1]{\ensuremath{\mathcal{O}(#1)}}

% Keywords

\newcommand{\Clousot}{\ensuremath{\mathtt{Clousot}}}

\newcommand{\code}[1]{\ensuremath{\mathtt{#1}}}

\newcommand{\takes}{\ensuremath{\leftarrow}}

\newcommand{\linea}[1]{\ensuremath{#1}}



% Syntax

\newcommand{\Assert}{\ensuremath{\mathtt{assert}}~}
\newcommand{\Assume}{\ensuremath{\mathtt{assume}}~}
\newcommand{\While}{\code{while}}
\newcommand{\If}{\code{if}}
\newcommand{\Else}{\code{else}}
\newcommand{\Skip}{\code{skip}}

\newcommand{\Stm}{\code{Stm}}
\newcommand{\Var}{\code{Var}}
\newcommand{\Exp}{\code{Exp}}
\newcommand{\BExp}{\code{BExp}}
\newcommand{\Bexp}{\BExp}
\newcommand{\ExpTwoOps}{\code{ExpTwoOps}}
\newcommand{\Lit}{\code{Lit}}
\newcommand{\Op}{\code{op}}
\newcommand{\Relop}{\code{relop}}
\newcommand{\Int}{\code{int}}

\newcommand{\Code}{\code{IstrStream}}
\newcommand{\Label}{\code{Label}}
\newcommand{\Istr}{\code{Istr}}
\newcommand{\Jump}{\code{jmp}}
\newcommand{\JumpIfTrue}{\code{jmpIf}}
\newcommand{\JumpIf}{\JumpIfTrue}
\newcommand{\Nop}{\code{nop}}

\newcommand{\Comp}{\ensuremath{\mathcal{C}}}
\newcommand{\Compexp}{\ensuremath{\mathcal{C}_{e}}}
\newcommand{\CompExp}{\Compexp}

\newcommand{\assign}{\el{assign}}
\newcommand{\test}{\el{test}}
\newcommand{\guard}{\test}
\newcommand{\checkif}{\el{check}}
\newcommand{\checkIf}{\checkif}

\newcommand{\aassign}{\el{\assign}}
\newcommand{\atest}{\el{\test}}

\newcommand{\arange}{\el{\mathsf{range}}}

\newcommand{\widening}{\ensuremath{\triangledown}}
\newcommand{\awidening}{\widening}
\newcommand{\narrowing}{\ensuremath{\vartriangle}}

\newcommand{\HLsem}[1]{\asemantica{H}{#1}}
\newcommand{\HLSem}[1]{\HLsem{#1}}
\newcommand{\LLsem}[1]{\asemantica{L}{#1}}
\newcommand{\LLSem}[1]{\LLsem{#1}}

\newcommand{\BigLLSem}[1]{\asemanticaBig{L}{#1}}
\newcommand{\bigLLSem}[1]{\BigLLSem{#1}}
\newcommand{\bigLLsem}[1]{\BigLLSem{#1}}

\newcommand{\result}{\code{res}}

% Definizioni per Subpolyhedra


\newcommand{\VarProg}{\ensuremath{\Var_\mathtt{P}}}
\newcommand{\VarSlack}{\ensuremath{\Var_\mathtt{S}}}

\newcommand{\variable}[1]{\ensuremath{\mathtt{#1}}}

\newcommand{\var}{\variable{v}}

\newcommand{\progvar}{\variable{x}}
\newcommand{\progvariable}[1]{\ensuremath{\progvar_{#1}}}

\newcommand{\slackvariable}[1]{\ensuremath{\beta_{#1}}}
\newcommand{\slackvar}{\slackvariable{}}

\newcommand{\slackvariableinfo}[1]{\ensuremath{\code{info}(#1)}}
\newcommand{\slackvarinfo}{\slackvariableinfo{\slackvariable{}}}


\newcommand{\reduction}[1]{\ensuremath{\rho(#1)}}
\newcommand{\simplify}[1]{\ensuremath{\sigma(#1)}}

\newcommand{\bottomS}{\ensuremath{\bottom_{S}}}
\newcommand{\topS}{\ensuremath{\top_{S}}}
\newcommand{\lessS}{\ensuremath{\aless_{S}}}
\newcommand{\joinS}{\ensuremath{\join_{S}}}
\newcommand{\wideningS}{\ensuremath{\widening_{S}}}

\newcommand{\intv}{\ael{i}}
\newcommand{\lineq}{\ael{l}}

\newcommand{\subpoly}{\ael{s}}

\newcommand{\subpolyPair}[2]{\ensuremath{\tupla{\lineq_{#1}^{#2};\ \intv_{#1}^{#2}}}}

%% Subsection trick to save some vertical space
\newcommand{\Fsubsubsection}[1]{\smallskip\noindent\textbf{#1}}


\newcommand{\Foxtrot}{\code{Foxtrot}}
\newcommand{\NET}{\code{.Net}}


\newcommand{\hint}[1]{\ensuremath{\mathbbm{h}_{#1}}}
\newcommand{\op}{\ensuremath{\diamond}}

\newcommand{\hintcomp}{\ensuremath{\Xi}}
\newcommand{\comment}[1]{}

\title{On the Relative Completeness of Bytecode Analysis versus Source Code Analysis} 
\author{Francesco Logozzo \& Manuel F\"ahndrich}

\institute{Microsoft Research\\
\email{\{logozzo,maf\}@microsoft.com} }


\begin{document}

\pagestyle{plain}

\maketitle

\begin{abstract}
We discuss the challenges faced by bytecode analyzers designed for
code verification compared to similar analyzers for source code.
While a bytecode-level analysis brings many simplifications, \eg,
fewer cases, independence from source syntax, name resolution, etc.,
it also introduces precision loss that must be recovered either via
preprocessing, more precise abstract domains, more precise
transfer functions, or a combination thereof.

The paper studies the \emph{relative} completeness of a static
analysis for bytecode compared to the analysis of the program
source.  We illustrate it through examples originating from the
design and the implementation of \Clousot{}, a generic static analyzer
based on Abstract Interpretation for the analysis of MSIL.
\end{abstract}

\section{Introduction}
We are interested in static program analysis for program verification,
where the goal is to infer \emph{invariants} that are
sufficient to discharge assertions which appear in the program either explicitly
(specified by the user through assertions) or implicitly
(\eg, array bound checks, null dereferences, division by zero,
etc.). Such analyses need to be precise enough to validate
the assertions.  In this paper, we will focus our attention on static
analyses for program verification and we call these PSA,
\emph{Precise enough Static Analyses}.

PSA are often designed to work at the program source level, \eg,
\cite{BlanchetCousotEtAl03,TVLA,Logozzo07,Bourdoncle93,Airac5}).
There are many reasons for that.  The program source provides a
uniform view which abstracts machine details.
Source code analysis is also able to directly exploit program
structure, such as loops, to increase the precision via techniques such
as reductive iterations~\cite{Granger92}, and the narrowing application by
re-execution from a post-fixpoint~\cite{CousotCousot77}. 

As we will see in this paper, the most immediate benefit of source
analysis however is that it provides the analysis designer with a
\emph{large code window}, allowing him/her to specialize transfer
functions for extra precision.

The analysis of low level code provides different
advantages:  1) it is more faithful, as it analyzes the code that is
actually executed (or closer to), 2) it enables the analysis
of libraries when source code is not available, 3) the analyzer
avoids redundant work that the compiler performed, such as name
resolution, type checking, template/generics instantiation,  4)
the semantics of high-level constructs that are expanded by the
compiler, such as $\mathtt{try\dots catch\dots finally}$, delegates,
partial classes in C\#, or generics in C\# and Java, need not be duplicated. As a consequence
a low-level code analyzer needs to deal with many fewer constructs
than a source analyzer, reducing its complexity.  Finally, 5) the analyzer can be
language independent; e.g., analyzing the common target
language MSIL of the .NET platform provides analysis of C\#, VB,
Managed C++, F\#.

Because of these advantages, plenty of static analyses have been
developed for low-level code.  Most of them address non-relational
properties like type checking \cite{MSIL,Leroy02,PalaczEtAl06}, non-cyclicity
\cite{RossignoliSpoto06}, nullness \cite{FahndrichLeino03}, etc.
Others target numerical properties, \eg, to check buffer overruns
\cite{BalakrishnanReps04} or array accesses
\cite{MendezNavasHermenegildo07}.

Our observation is that while writing a static analyzer for a
low-level language or bytecode is simpler than writing one for source
code due to the above advantages, it is non-trivial to match the
precision of a similar analysis performed at source level, due to the
missing high-level structure and the reduced size of the \emph{code
  window} used by transfer functions. The rest of
this paper elucidates this observation with examples and general
principles.

\begin{example}[Motivating Example]
\label{ex:intro}
Suppose we analyze a program containing the high level statement
$\mathcal{S} \equiv \Assume \code{x} - \code{y} \leq 7$, using the
difference bounds abstract domain~\cite{Mine01-1}.  At source level,
the constraint $\code{x} - \code{y} \leq 7$ is a difference
constraint, and it can be represented faithfully by the abstract
state.  Now consider the compilation of $\mathcal{S}$ into three
address code:
\[
\begin{split}
0: & \code{t_1} \takes \code{x} - \code{y} \\
1: & \code{t_2} \takes \code{t_1} \leq 7 \\
2: & \Assume \code{t_2}
\end{split}
\]   
Analyzing this code sequence with the same domain used at the source
level raises immediate problems:
\begin{description}
\item[Expression complexity]
The assignment at line \linea{0} involves three variables, which
cannot be captured precisely by the difference bounds domain. As a
consequence, the abstract value for \code{t_1} is $\top$.

\item[Type complexity] At line \linea{1}, \code{t_2} is assigned the
result of a boolean expression\footnote{Please note that this case is
orthogonal to the previous one, \ie, the problem shows up even if the
assignment was $ \code{t_2} \takes (\code{x} - \code{y} \leq 7)$.}.
At the source level there was no such boolean assignment, and in fact,
the domain used at source level cannot encode the relation between
\code{t_1} and \code{t_2}.
\end{description}
As a result, the analysis of the code sequence using the same domain
as at the source level produces an abstract state that contains no
information about the relation of \code{x} and \code{y}.
Several solutions are possible to mitigate the above problems.
\begin{itemize}
\item Use a more precise numerical abstract domain for the low-level
analysis that handles relations among more than two variables, such
as Octahedra \cite{ClarisoCortadella04}, or Polyhedra
\cite{CousotHalbwachs78,PPL}. This approach however leads to
scaling problems, as these domains exhibit exponential complexity. No
polynomial domains are known that can handle more than two variables \cite{Mine04}.

\item Split the current abstract domain in two at the boolean
assignment: one where $\code{t_2} == \code{true}$ and one where
$\code{t_2} == \code{false}$.  This method has two main drawbacks: (i)
it may lead to exponential explosion by doubling the abstract states
at each conditional branch; and (ii) it still introduces loss of
precision, because the relation to be assumed at line \linea{2} is
lost.

\item 
A more general solution which addresses both of the problems and all
others related to the limited code window, is to use a
lightweight symbolic abstract domain to compute available
expressions at each program point. 
\end{itemize}
\vspace{-3pt}
Let us briefly sketch how the use of a symbolic domain to recover
expressions works on the example.  At line \code{2}, the analysis
\emph{first} asks the symbolic domain to refine variable \code{t_2}.
This refinement, using line \linea{1}, produces $\code{t_2} \equiv
\code{t_1} \leq 7$, which can be further refined, using line \linea{0},
to produce $\code{t_2} \equiv \code{x} - \code{y} \leq 7$.  The
analysis \emph{then} passes the refined expression $\code{x} -
\code{y} \leq 7$ to the difference bounds domain, which handles it
exactly as the source analysis does.  \qed
\end{example}

\vspace*{-5pt}
As the example shows, PSA of low-level code requires more than just
reusing the domains suitable for high-level code, otherwise, precision
is lost. In this paper, we investigate the \emph{relative}
completeness of low-level code analysis versus source code
analysis, i.e., what is required for bytecode analysis to be as
precise as source code analysis, without requiring the use of domains
with worse complexity.

We present representative issues that crop up when designing precise
and scalable bytecode analyses. We faced those issues during the
design and implementation of \Clousot{} \cite{LogozzoMaf08}, a PSA for .NET based on
abstract interpretation.  The issues described are not specific to
.NET, but arise for all low-level analyses. They manifest in (i) the
precise handling of assignments, tests and branches, and (ii) the
fixpoint iteration strategy, in particular for narrowing and reductive
iterations.  We discuss how to overcome these issues, and the
solutions we have adopted in \Clousot{}.
%%%
In general, quantifing the impact of such issues is hard.
We tried a rough (under-)estimation by switching off some precision refinements discussed in this paper (not all of them could be switched off, as many are buried deep in the architecture of \Clousot{}).  
We obtained a loss of precision of 10\% in the analysis of the array accesses of \code{mscorlib.dll}, the main library in the .NET framework.
Such loss of precision is enough to generate more than 1400 false positives, \ie, to make the analysis \emph{de facto} unuseful.

\vspace{-10pt}

\section{Languages}
We use a while-language as a representative for high-level languages,
and a three address code instruction set as a
representative of low-level code.

\vspace{-13pt}
\subsection{While-Language}

\begin{figure}[t]
\vspace*{-10pt}
\[
  \begin{array}{rcll}
    \Stm &::=&  \Skip; \mid \Var := \Exp; \mid \Stm~ \Stm \mid \While(\Bexp)~\{\Stm\};  \mid \If(\Bexp)~ \{ \Stm~ \} \Else~\{ \Stm\ \}; \mid \\
&& \Assume \BExp; \mid \Assert \BExp; \\
    \Exp & ::= & \Lit \mid \Exp~ \Op~ \Exp   \\
    \BExp & ::= & \Lit \mid \Exp~ \Relop~ \Exp \mid ~!(\BExp) \mid  \Bexp~ \&\&~ \Bexp \mid \Bexp \parallel \Bexp \\
    \Lit & ::= & \Var \mid \Int   \quad \quad \quad  \quad \Var  :: = \dots \mid \code{x} \mid \code{y} \mid \dots \quad \quad \quad     \Int  :: =  \dots \mid -1 \mid 0 \mid 1 \mid \dots \\
   \Op & ::= & + \mid - \mid * \mid /\quad \quad \quad     \Relop  ::=  < \mid \leq \mid ==   
  \end{array}
\]
\vspace*{-10pt}
  \caption{The while-language: a high-level language}
  \label{fig:WhileLanguage}
\end{figure}

Our high level language is a simple while-language with no dynamic
memory allocation, shown in Fig.~\ref{fig:WhileLanguage}.  The
semantics is standard.  We use
a single type, integers.  Following widespread convention, we
assume that \code{0} stands for \code{false} and all the other
integers for \code{true}.  
\comment{
So, for instance:
\[
\begin{array}{ll}
& \sem{ \If(0)~ \Stm_1~ \Else~ \Stm_2;  } = \sem{\Stm_2} \\
\text{ and } & \sem{ \If(2 \leq 12 )~ \Stm_1~ \Else~ \Stm_2;  } =  \sem{ \If(1)~ \Stm_1~ \Else~ \Stm_2; } = \sem{\Stm_1}.
\end{array}
\]
}
Boolean expressions \emph{shortcut} evaluation.
\comment{
\[
\begin{array}{ll}
  & \sem{0~ \&\& 1/0} = 0 \text{ and }  \sem{37 \parallel 1/0} = 1 \\
\text{ but } & \sem{1/0~ \&\& 0} =  \sem{1/0 \parallel 37} = \textit{"error: division by zero"}. 
\end{array} 
\]
}
We also consider assert and assume statements, which enable assume/
guarantee reasoning, \eg, to (abstract) method calls.  The
statement $\Assert \code{e};$ checks if the expression \code{e}
holds. If it does not, then the program fails.
\comment{
\[
\sem{\code{x} := 0; \Assert 0 < \code{x}; \code{y} := 1/\code{x}} = \textit{"error: assertion violated"}.
\]
}
The statement $\Assume \code{e};$ acts as an execution guard for the
following statements. If the condition does not hold, execution gets stuck. 
\comment{
For instance in the code snippet below, the assumption acts as
a guard to the division, so that if $0 \geq \code{x}$, then the
program simply stops.
\[
\sem{\code{x} := 0; \Assume 0 < \code{x}; \code{y} := 1/\code{x}} = \textit{"stop: assumption not satisfied"}.  
\]
}

\subsection{Three Address Code}
Our low-level language is a three address code instruction set shown
in Fig.~\ref{fig:ThreeAddressesInstructionSet}.  This
language is higher level than MSIL, Java bytecode, or assembly, but it
simplifies our presentation and is sufficient to exhibit the problems
of interest.
\begin{figure}[t]
\vspace*{-10pt}
  \[
  \begin{array}{rcll}
    \Code &::=&  \Label : ~\Istr \mid \Label :~ \Istr~ '\backslash n'~ \Code \mid \epsilon \\
    \Label & ::= & 0 \mid ... \mid 2^{32} \\
    \Istr & ::= &\Var \takes \ExpTwoOps \mid \\ 
    && \Jump~ \Label \mid \JumpIfTrue~ \Var~ \Label \mid  \Assert \Lit \mid \Assume \Lit \mid \Nop \\
    \ExpTwoOps & ::= &  \Lit \mid  \Lit~ \Op~ \Lit  \mid \Lit~ \Relop~ \Lit \mid  \Lit~ \&\&~ \Lit \mid \Lit \parallel \Lit  
  \end{array}
  \]
\vspace*{-10pt}
  \caption{Three address code: a low-level language.}
  \label{fig:ThreeAddressesInstructionSet}
\end{figure}

An instruction stream is a sequence of labeled instructions.  An
assignment instruction $\code{x} \takes \code{e_{2ops}}$ updates the
value of the variable \code{x} with the result of the evaluation of
the expression $\code{e_{2ops}}$ which contains \emph{at most} two
operands.  As a consequence the expressions that can be atomically
evaluated and assigned at low level are a subset of those at higher
level, \ie, $\ExpTwoOps \subseteq \Exp \cup \BExp$.  In the next
sections, we will see how this impacts the precision and performances
of PSA.


\subsection{Compilation}
We assume two compilation functions: $\Comp \in
\funzione{\Stm}{\Code}$ compiles a program expressed in the
high-level language into a low-level instruction stream one, and $\Compexp \in
\funzione{(\Exp \cup \Bexp)}{\Code}$ compiles expressions into a
sequence of instructions for evaluating them.  The result of the
evaluation is in a (reserved) variable \result.
We expect the functions \Comp\ and \Compexp\ to perform
naive compilation, \ie, a straightforward translation without any
program optimization \cite{AhoSethiUllman}.
\nopagebreak[4]
\section{Abstract Interpretation}
Abstract interpretation is a theory of approximations
\cite{CousotCousot77}.  It formalizes the intuition that semantics are
more or less precise depending on the observation level.  The more
precise the abstract semantics, the more precise the properties about
the execution of the program it captures.  A static analysis is an
abstract semantics which is rough enough to be computable.  A precise
static analysis is a static analysis which is precise \emph{enough} to
capture the properties of interests, \eg, those needed to prove the
absence of certain runtime errors.

\subsection{Abstract Domains}
%relational, non-relational, numerical (intervals, octagons, polyhedra)
An abstract domain \adom{D} is the complete lattice
\tupla{\el{E},\aless, \abot, \atp, \acup, \acap}, where \el{E} is the
set of abstract elements, ordered according to the relation $\aless$. 
The smallest abstract element is $\abot$, the largest is $\atp$. 
The join \acup, and the meet \acap,  are also defined.
With a slight abuse of notation, we will confuse an  abstract domain \adom{D} with the set of its elements \el{E}.

The elements of an abstract domain are related to the concrete domain
\dom{D} (also a complete lattice), by means of a monotonic
concretization function $\gamma \in \funzione{\adom{D}}{\dom{D}}$.
\comment{
\footnote{We do not require a best abstraction function
$\alpha$, thus putting ourselves in a more generic framework than the
one based on Galois connections,~\cite{CousotCousot92-2}.}. 
}
In this
paper we assume the concrete domain to be the complete boolean lattice
$\parti{\Sigma}$, where $\Sigma = \funzione{\Var}{\mathbb{Z}}$.

Given two abstract domains, $\adom{D}_1$ and $\adom{D}_2$, their reduced cartesian product  is $\adom{D}_1 \otimes \adom{D}_2$, whose elements are pairs which satisfy the reduction condition: 
\[
\forall \tupla{\ael{d}_1, \ael{d}_2} \in \adom{D}_1 \otimes \adom{D}_2.\ \gamma_{\adom{D}_1 \otimes \adom{D}_2}(\tupla{\ael{d}_1, \ael{d}_2}) \subseteq \gamma_{\adom{D}_1}(\ael{d}_1) \cap \gamma_{\adom{D}_2}(\ael{d}_2) \ .
\]

An abstract domain is said to be \emph{relational} if it keeps
relations between program variables.  Otherwise it is said to be
\emph{non-relational}.

The elements of the abstract domain of intervals, \Intervals, are $\{
[i, s] \mid i,s \in \mathbb{Z} \cup \{-\infty, + \infty\} \}$.  
\comment{
The
order is the interval inclusion, the bottom element is the empty
interval (\ie, an interval where $s < i$), the largest element is the
line $[-\infty, +\infty]$, the join and the meet are respectively the
union and the intersection of intervals.  
}
The concretization function,
$\gamma_\Intervals \in \funzione{\Intervals}{\parti{\mathbb{Z}}}$ is
defined as $\gamma_\Intervals([i, s]) = \{ z \in \mathbb{Z} \mid i
\leq z \leq s\}$.  The abstract domain of boxes, \Boxes, is the
functional lifting of \Intervals, \ie, $\Boxes =
\funzione{\Vars}{\Intervals}$.  The concretization of a box,
$\gamma_\Boxes \in \funzione{\Boxes}{\parti{\Sigma}}$ is defined
as $\gamma_\Boxes(f) = \{ \sigma \in \Sigma \mid \forall
\code{x}. \code{x} \in \mathrm{dom}(f) \Longrightarrow
\sigma(\code{x}) \in \gamma_\Intervals(f(\code{x}))\}$.  From the
definition of $\gamma_\Boxes$, it follows that the meaning of a
variables in \Boxes\ is independent from all the others, which implies
that \Boxes\ is a non-relational abstract domain.  The time and space
complexity of the operations on \Boxes\ is \Order{n}, where $n$ is the number
of variables.

\comment{
The abstract domain of difference bounds matrices, \DBM~\cite{Mine01-1}, captures properties in the form of $\code{x} - \code{y} \leq z$, with $ z \in  \mathbb{Z}$.
The abstract domain of octagons, \Octagons~\cite{Mine01-2}, refines \DBM\ by capturing properties in the form of $\pm \code{x} \pm \code{y} \leq z$, with $z \in \mathbb{Z}$.
Octagons elements are represented as a weighted direct graph, and the lattice operations are defined point-wisely.  
The meaning of octagon elements is given by the concretization $\gamma_\Octagons \in \funzione{\Octagons}{\parti{\Sigma}}$ defined as $\gamma_\Octagons(o) = \bigcap_{\code{x} \stackrel{z}{\rightarrow} \code{y} \in o} \{ \sigma \in \Sigma \mid \sigma(\code{x}) - \sigma(\code{y}) \leq z\}$.
From the definition of $\gamma_\Octagons$ it follows that \Octagons\ can capture relations between \emph{two} variables, thus it is a (weakly) relational domain.
\Octagons\ is the most precise known numerical abstract domain which shows a polynomial complexity: it is \Order{n^3} in time  and \Order{n^2} in space.
}

The abstract domain of Polyhedra, \Polyhedra
~\cite{CousotHalbwachs78}, captures linear constraints between program
variables: $\sum_{i=0}^{i < n} a_i * \code{x}_i \leq z$, with $a_i, z
\in \mathbb{Z}$.
\comment{
  Elements have an internal double representation (as
system of inequalities and as a set of vertexes and generators).  It
is possible to move from one representation to the other via an
expensive conversion.  Lattice operations are defined on one or the
other representation.  
}
The concretization function $\gamma_\Polyhedra
\in \funzione{\Polyhedra}{\parti{\Sigma}}$ is defined as the intersection of all the constraints :
$\gamma_\Polyhedra(P) = \bigcap_{\sum_{i=0}^{i < n} a_i * \code{x}_i
\leq z \in P} \{ \sigma \in \Sigma \mid \sum_{i=0}^{i < n} a_i *
\sigma(\code{x}_i) \leq z \}$.  From the concretization function, it
follows that \Polyhedra\ can capture properties between an arbitrary
number of variables, thus it is a relational domain.  The complexity
of \Polyhedra\ is \Order{2^n} both in space and time.
\comment{
In practice, program analysis is about finding abstract domains which
represent a good trade-off between precision and cost.
}

\subsection{Transfer Functions}
\comment{
Let \dom{D} be a concrete domain, \adom{D} an abstract domain related
to the concrete by a Galois connection: $\dom{D}
\galois{\alpha}{\gamma} \adom{D}$, and $\tau \in
\funzione{\dom{D}}{\dom{D}}$ be a transfer function.  Then, by a basic
result in abstract interpretation theory, $\ael{\tau}^* \in
\funzione{\adom{D}}{\adom{D}}$ defined as $\ael{\tau}^* = \alpha \circ
\tau \circ \gamma$ is the best \emph{abstract} transfer function.  The
definition of $\ael{\tau}^*$ cannot be used in practice in the
implementation of PSA.  In general $\ael{\tau}^*$ is not computable,
and when it is, it may be too expensive to be adopted in
practice,~\cite{Yorsh04}.  Furthermore, it may be the case that
$\ael{\tau}^*$ simply does not exist.  For instance, for \Polyhedra,
there exists no best abstraction function
$\alpha$,~\cite{CousotCousot92-2}, and consequently no $\ael{\tau}^*$.
}

Abstract interpreters implement an upper approximation $\ael{\tau}$ of
the best abstract transformer $\ael{\tau}^*$, \ie~ $\forall
\ael{d}\in\adom{D}.~ \ael{\tau}^*(\ael{d}) \aless
\ael{\tau}(\ael{d})$.  An abstract transfer function $\ael{\tau}$ is (i)
usually hand-crafted, and (ii) tuned to maximize the precision/cost trade-off.

It is common practice for the implementation of an
abstract domain \adom{D} to provide two abstract transfer functions:
one for the assignment and one for the handling of
tests~~\cite{BlanchetCousotEtAl03,Logozzo07,Venet04}.  The
assignment abstract transfer function, \adom{D}.\aassign, is an
over-approximation of the states reached with the concrete assignment:
\[
\forall \code{x}, \code{e}. \forall \ael{d}.\ \{ \sigma[\code{x} \mapsto v] \mid \sigma \in \gamma(\ael{d}), \semantica{}{\code{e}}(\sigma) = v \} \subseteq \gamma(\adom{D}.\aassign(\ael{d}, \code{x}, \code{e})).
\]
The test abstract transfer function, \adom{D}.\atest, acts as a kind of filter to the input states:
\[
\forall \code{e}. \forall \ael{d}.\ \{ \sigma \in \gamma(\ael{d}) \mid \semantica{}{\code{e}}(\sigma) \neq 0 \} \subseteq \gamma(\adom{D}.\atest(\ael{d}, \code{e})).
\]
It is vital for a PSA to provide a precise approximation of \atest. 

\comment{
\subsection{Widening and Narrowing Operators}
Most of the abstract domains used in practice do not satisfy the ascending chain condition (ACC), so that the least fixpoint computation on such domains may not terminate.
A widening operator is then used to extrapolate the sequence limit.
Stated otherwise, it enables for \emph{dynamic} approximation.
Formally, a widening operator $\awidening \in \funzione{\adom{D} \times \adom{D}}{\adom{D}}$ is such that 
$\forall \ael{d}_1, \ael{d}_2 \in \adom{D}.\ \ael{d}_1 \aless \ael{d}_1 \awidening \ael{d}_2 \text{ and } \ael{d}_2 \aless \ael{d}_1 \awidening \ael{d}_2$ and
for all the increasing chains $\ael{d}_0 \aless \dots \ael{d}_n \aless \dots$ the increasing chain defined as $\ael{w}_0 = \ael{d}_0, \dots \ael{w}_{i+1} = \ael{w}_{i} \awidening \ael{d}_{i+1}$ is not strictly increasing.
Then, the upward fixpoint iterations with widening will converge to a post-fixpoint, \cite{CousotCousot77}.

Precision lost by the application of the widening can be recovered with a narrowing operator.
A narrowing operator essentially refines the computed post-fixpoint, without going below the least fixpoint, \ie, by preserving soundness.
Formally, a narrowing operator is $\narrowing \in \funzione{\adom{D} \times \adom{D}}{\adom{D}}$ such that $\forall \ael{d}_1, \ael{d}_2.\ \ael{d}_1 \aless \ael{d}_2 \Longrightarrow \ael{d}_1 \aless (\ael{d}_2 \narrowing \ael{d}_1)\aless \ael{d}_2$ and for all the decreasing chains $\ael{d}_0 \asup \dots \ael{d}_n \asup \dots$ the increasing chain defined as $\ael{n}_0 = \ael{d}_0, \dots \ael{n}_{i+1} = \ael{n}_{i} \narrowing \ael{d}_{i+1}$ is not strictly decreasing.
Then, the downward fixpoint iterations with narrowing will provide a more precise yet sound approximation of the least fixpoint, \cite{CousotCousot77}.

The design of good widening and narrowing operators is very important for a PSA.
}

\section{Relative Completeness of Precise Analysis of Bytecode}
In this section, we define a generic abstract semantics for the high
level language, $\HLsem{\cdot} \in \funzione{\Stm}{\adom{D}
\rightarrow \adom{D}}$, by structural induction.  In parallel, we
define the abstract semantics for the low level language,
$\LLsem{\cdot} \in \funzione{\Code}{\adom{D} \rightarrow
  \adom{D}}$. For each kind of statement and expression, we
(i) express whether and under what conditions
$\LLsem{\cdot}$  is complete w.r.t.\  $\HLsem{\cdot}$, \ie, when
$\LLSem{\cdot}$ is as precise as $\HLSem{\cdot}$, and (ii) show
how best to overcome precision problems, \eg, by refining the abstract domain or
the transfer functions.

\subsection{Notions of Relative Completeness}
We distinguish two notions of relative completeness: strong and
weak.   Strong relative completeness requires the low-level
analysis not to lose information when using the \emph{same} abstract domain.
Weak relative completeness allows the low-level analysis to use a
\emph{refinement} of the abstract domain used at source level.

\begin{definition}[Strong Relative Completeness]
Given statement \Stm, abstract domain \adom{D}, and projection
function $\pi \in \funzione{\adom{D}}{\adom{D}}$, which removes all
the temporary variables introduced by compilation, if
\begin{equation}
\forall \ael{d} \in \adom{D}.\ \pi(\LLSem{\Comp(\Stm)}(\ael{d})) \aless \HLSem{\Stm}(\ael{d}),
\label{eq:completeness}
\end{equation}
 then $\LLSem{\cdot}$ is strong-relatively complete w.r.t. to $\HLSem{\cdot}$ for statement $\Stm$.
\end{definition}

Note that the definition above does not require equality of precision,
only subsumption. It may be the case that the analysis at the bytecode
level is more precise in some cases.

\begin{definition}[Weak Relative Completeness]
Given statement \Stm, two abstract domains \adom{D} and $\adom{D}^+$
such that $\adom{D}^+$ is more precise than $\adom{D}$ : $\adom{D}^+
\galois{\alpha}{\gamma} \adom{D}$, and projection function $\pi \in
\funzione{\adom{D}^+}{\adom{D}^+}$, which removes all the temporary
variables introduced by compilation, if
\begin{equation}
\forall \ael{d} \in \adom{D}.\ \alpha (\pi (\LLSem{\Comp(\Stm)}(\gamma(\ael{d})))) \aless \HLSem{\Stm}(\ael{d}),
\label{eq:weakcompleteness}
\end{equation}
 then  $\LLSem{\cdot}$ is weak-relatively complete
 w.r.t. to $\HLSem{\cdot}$ for statement $\Stm$ up to the refined
 domain $\adom{D}^+$.
\end{definition}

Weak relative completeness relaxes the previous definition by enabling
the use of a more precise abstract domain for the analysis of the
bytecode.  It is evident that strong relative completeness
implies weak relative completeness.

\subsection{Skip}
Handling of \Skip\ is straightforward:
$\HLsem{\Skip} = \lambda \ael{d}.\ael{d}$.
The \Skip\ statement is compiled with a \Nop: $\Comp(\Skip) = \mathit{n}: \Nop$, and $\LLsem{\mathit{n}: \Nop} = \lambda \ael{d}.\ael{d}$.
As a consequence, in this case the bytecode analysis is trivially strongly complete. 

\subsection{Sequence}
The analysis of a sequence of statements is usually just the composition of the analyses:
\begin{equation}
\HLSem{\Stm_1 \Stm_2} = \HLSem{\Stm_2} \circ \HLSem{\Stm_1}.
\label{eq:seqHL}
\end{equation}
The compilation is the juxtaposition of two sequences of instructions: 
\[
\Comp(\Stm_1 \Stm_2) = \left [ \begin{array}{ll}\Comp(\Stm_1)\\ \Comp(\Stm_2) \end{array} \right. \ .
\]
The abstract semantics of a sequence of instructions is the compositions of the analyses: 
\begin{equation}
\LLSem{\mathit{k}: \Istr\ '\text{\textbackslash} \mathit{n}'\ \Code} = \LLSem{\Code} \circ \LLSem{\mathit{k}: \Istr}.
\label{eq:seqLL}
\end{equation}
Assuming that low-level analysis is complete (resp. weakly complete)
for the subsequences, from (i) the fact that projection is an abstraction; and (ii) the monotonicity of the abstract functions, it follows that
the low-level analysis of the sequence is complete (resp. weakly complete) w.r.t. the high-level analysis.

Note that in general, sequencing may cause loss of precision for both
high- and low-level analysis w.r.t. the concrete semantics.

\comment{
\subsubsection{Discussion: Completeness w.r.t. the Concrete Semantics.}
It is worth noting that in general the definition of abstract
semantics for composition as in (\ref{eq:seqHL}) and (\ref{eq:seqLL})
is an incomplete abstraction of the \emph{concrete} semantics of the
sequential composition as shown by the next example.

\begin{figure}[t]
  \[
  \begin{array}{ll}
    \Stm_1 & 
    \left[ 
      \begin{array}{l}
        \If~ ( 0 < \code{y})  \\
        \quad  \{ \code{x} := -1; \} \\
        \Else~ \\
        \quad \{ \If~ ( \code{y} < 0 ) \\
        \quad \quad \{ \code{x} := 1; \} \\
        \quad \phantom{\{} \Else~ \{~ \Skip;~ \} \};   
      \end{array}
    \right.\\
    \Stm_2 \ & 
     \left[ 
      \begin{array}{l}
        \code{z} := \code{x} * \code{y}; 
      \end{array}
    \right.
  \end{array}
  \]
\caption{An example showing the lose of precision in handling the sequential composition.}
\label{fig:exSeq}
\end{figure}

\begin{example}[Incompleteness of Sequential Composition]
Consider the code snippet in Fig.~\ref{fig:exSeq}, and  suppose we analyze it with the \Boxes\ domain.
The separate analysis of $\Stm_1$ and $\Stm_2$ is complete, in that 
$ \alpha_\Boxes(\semantica{}{\Stm_1}(\Sigma)) =  \HLSem{\Stm_1}(\atp_\Boxes) = [\code{y} \mapsto [-\infty, +\infty], \code{x} \mapsto [-1, 1]] = \ael{b}_1
$ 
and 
\[
\alpha_\Boxes(\semantica{}{\Stm_2}(\gamma_\Boxes(\ael{b}_1))) =  \HLSem{\Stm_2}(\ael{b}_1) =  \ael{b}_1[\code{z} \mapsto [-\infty, +\infty]] =  \HLSem{\Stm_1 \Stm_2}(\atp_\Boxes)
\] 
On the other hand, 
$
\semantica{}{\Stm_1 \Stm_2}(\Sigma) = \{ \sigma \in \Sigma \mid -1 \leq \sigma(\code{x}) \leq 1, \sigma(\code{z}) \leq 0 \},
$
 so that the abstraction of the sequence is incomplete:
$
\alpha_\Boxes(\semantica{}{\Stm_1 \Stm_2}) = [\code{x} \mapsto [-1, 1], \code{z} \mapsto [-\infty, 0]] \ael{\sqsubset}  \HLSem{\Stm_1 \Stm_2}(\atp_\Boxes).
$ 
\qed
\end{example}

The example highlights that splitting a code sequence and analyzing
each part separately (even at source level) may lead to precision loss.  In
general, this issue can be addressed by enlarging the granularity of
the transfer function. Several techniques for coping with this problem
have been proposed, \eg, trace partitioning \cite{Tzolovski98},
disjunctive completion \cite{CousotCousot92-1}, local refinement
\cite{ColbyLee96}, systematic domain refinement \cite{Giacobazzi00},
problem-specific optima \cite{Monniaux07}. These work at the bytecode
level as well.
}

\subsection{Assignments}
\label{sec:assignment}
A source language analysis just passes the assignment to the underlying abstract domain \adom{D}: 
\begin{equation}
\HLsem{\code{x}:= \code{e};} = \lambda \ael{d}. \adom{D}.\aassign(\ael{d}, \code{x}, \code{e}).
\label{eq:assignHL}
\end{equation}
The compilation of the assignment generates a sequence of instructions
to evaluate \code{e}, and an assignment of the result to \code{x}: 
\begin{equation}
\Comp(\code{x}:= \code{e};) = \left [ \begin{array}{ll}& \Compexp(\code{e}) \\ \mathit{k} :& \code{x} \takes \result\end{array} \right. \ .
\label{eq:assignCompile}
\end{equation}
Without loss of generality, we will assume in the sequel that the last
instruction of $\Compexp(\code{e})$ assigns directly to the target
variable \code{x} instead of \code{res}.
Thus, the final assignment is similarly passed to underlying abstract domain:
\begin{equation}
\LLsem{\mathit{k}: \code{x} \takes \code{e2op}} = \lambda \sigma. \adom{D}.\aassign(\sigma, \code{x}, \code{e2op}).
\label{eq:assignLL}
\end{equation}
If the source expression \code{e} is such that $\code{e} \equiv \code{l}$ or $\code{e} \equiv \code{l}_1 \Op\ \code{l}_2$, where $\code{l}, \code{l}_1, \code{l}_2 \in \Lit$, and $\Op$ is as in Fig.~\ref{fig:WhileLanguage}, then (\ref{eq:assignHL}), (\ref{eq:assignCompile}) and (\ref{eq:assignLL}) imply the strong relative completeness of \LLsem{\mathit{k}: \code{x} \takes \code{e2op}}.
However, this is not the case for more complex expressions, as the next ({counter-}) examples show.
\comment{
\begin{figure}[t]
  \[
  \begin{array}{r | r | l c r}
    \code{x} & \code{y} &\phantom{0} (\code{x} + \code{y}) * \code{y}  \\   
    2 &\phantom{0} -1 &\phantom{0} (2-1)* -1 & = & -1 \\
    2 & 1 &\phantom{0} (2 +1 ) * 1 & = & 3 \\
    3 & -1 &\phantom{0} (3 -1)*-1 & = & -2 \\
    3 & 1 &\phantom{0} (3 + 1) * 1 & = & 4
  \end{array}
  \]
  \caption{The tabular evaluation of $(\code{x} + \code{y}) * \code{y}$ in the abstract state  $\ael{b}_0 = [\code{x} \mapsto [2, 3], \code{y} \mapsto [-1, 1]]$. The table is for presentation purposes only, in practice an implementation will avoid explicitly computing the table, by using some smarter representation, ~\cite{Logozzo07}.}
  \label{fig:ex1}
\end{figure}
}
\begin{example}[Precision Loss using Interval Arithmetic]
\label{ex:intervalarithmetics}
Suppose we use the \Boxes\ domain to analyze the assignment $\mathcal{A} \equiv \code{z} := (\code{x} + \code{y}) * \code{y}$.
Let $\ael{b}_0 = [\code{x} \mapsto [2, 3], \code{y} \mapsto [-1, 1]]$ be the abstract input state.
Then
\[
\HLSem{\code{z} := (\code{x} + \code{y}) * \code{y};}(\ael{b}_0) = \ael{b}_0[\code{z} \mapsto [-2, 4]],
\]
using a specialized source transfer function.
\comment{as the right-hand side expression evaluates according to the table in Fig.~\ref{fig:ex1}.}
On the other hand, the compilation of $\mathcal{A}$ is 
\begin{equation}
  \Comp(\code{z} := (\code{x} + \code{y}) * \code{y};) = \left [
    \begin{array}{r l}
      0: & \campo{t} \takes \campo{x} + \campo{y} \\
      1: & \campo{z} \takes \campo{t} * \campo{y}
    \end{array} \right. ,
  \label{ex:comp1}
\end{equation}
so that the abstract state after the program point $0$ is
$\ael{b}_0[\campo{t} \mapsto [1,4]]$, and hence the abstract
post-state is $\LLSem{\Comp(\code{z} := (\code{x} + \code{y}) *
\code{y};)} (\ael{b}_0) = \ael{b}_0[\campo{t} \mapsto[1,4], \campo{z}
\mapsto [\mathbf{-4}, 4]]$. \qed
\end{example}
The example shows that the analysis of the compiled code introduces a
loss of precision w.r.t. to a specialized source level transfer
function.  Intuitively, it is caused by the fact that the domain
\Boxes\ is non-relational, and hence at program point $1$ it has lost
the information that \code{t} depends on \code{y}, so that two
spurious cases are introduced.

As the incompleteness originates from the use of a non-relational
numerical domain, one may advocate the usage of a relational domain.
If we chose to analyze (\ref{ex:comp1}) with \Octagons, the problem,
unfortunately, does not go away.
At program point $0$, we have an assignment that involves \emph{three} variables.
The domain cannot track the relation between \code{t}, \code{x} and \code{y}.
As a consequence, no improvement is obtained at $1$ using Octagons.

If we chose instead to analyze (\ref{ex:comp1}) with \Polyhedra, then
the assignment at $0$ can be precisely captured by this domain.  So
the abstract post-state is $\ael{p} = \{ 2 \leq \code{x} \leq 3, -1
\leq \code{x} \leq 1, \code{t} - \code{x} -\code{y} = 0\}$.  
The instruction at $1$ involves a quadratic expression (the
multiplication of two variables), which a naive implementation of
\Polyhedra.\aassign\ may simply decide to ignore.  However, it is easy
to see how a more refined implementation can figure out that, because
of $\ael{p}$, $\code{t} = \code{x} + \code{y}$ it can use this
equality to simplify the multiplication and infer the tightest lower
bound $-2 \leq \code{z}$, and hence satisfy
(\ref{eq:weakcompleteness}).


\begin{example}[Precision Loss using Octagons]
Let us analyze the assignment $\mathcal{B} \equiv \code{z} := 2
*\code{x} - \code{y};$ with the \Octagons\ domain.  Let the initial
abstract state be $\ael{o}_0 = \{ \code{x} -\code{y} \leq 1, \code{y}
- \code{x} \leq -1 \}$.  Even if the source expression is not in the
octagonal form, the designer of the domain can refine
\Octagons.\aassign\ (i) to replace \code{x} in the right hand side of
the $\mathcal{B}$ by $\code{y} - 1$, and (ii) to perform the basic
algebraic simplifications, so that
\[
\HLSem{\code{z} := 2 *\code{x} - \code{y};}(\ael{o}_0) = \ael{o}_0 \cup \{ \code{z} - \code{y} \leq 2, \code{y} - \code{z} \leq -2 \}.
\]
On the other hand, the compilation of $\mathcal{B}$ is 
\begin{equation}
  \Comp(\code{z} := 2*\code{x} - \code{y};) = \left [
    \begin{array}{r l}
      0: & \campo{t} \takes 2 * \campo{x} \\
      1: & \campo{z} \takes \code{t} - \code{y}
    \end{array} \right. .
  \label{ex:comp2}
\end{equation}
\end{example}
At program point $0$, there is no way one can refine
\Octagons.\aassign\ to provide an octagonal constraint for \code{t}.
For instance, the substitution of \code{x} by $\code{y} -1$ produces
$\code{t} \takes 2 * \code{y} - 2$, which cannot be represented by an
octagon constraint, too.  As a consequence, no constraint can be
inferred on \code{t} and hence \code{z}: $\LLsem{\Comp(\code{z} := 2
*\code{x} - \code{y};)}(\ael{o}_0) = \ael{o}_0$. \qed

Intuitively, the precision loss in the previous example is caused by
splitting ``large'' expressions into smaller chunks, thereby
reducing the expression window seen by the atomic operations in the
abstract domain, and hence limiting their ability to infer relations.

If we chose instead to analyze (\ref{ex:comp2}) with \Polyhedra, then
both assignments at program points $0$ and $1$ are linear
constraints that are represented exactly by this abstract domain.
As a consequence, the low-level analysis, when performed on a more
precise abstract domain is (weak-relatively) complete.

\subsubsection{Discussion: Choosing the Right Abstract Domain.}
The previous examples suggest that we can obtain weak completeness by
systematically using \Polyhedra. This is the direction taken by some
analyzers for low-level code, \eg,
\cite{GopanReps06,MendezNavasHermenegildo07,boogie}. We do not
advocate this approach, as \Polyhedra{} exhibits an exponential
complexity in practice (in the number of variables).  In order to
overcome this issue in \Clousot{}, we have chosen to not refine
directly the numerical domain \adom{D}, but to combine it with a
symbolic domain \Symbolic\ to propagate expressions,
\cite{AhoSethiUllman,Mine06}. 
In other words the analysis is done on the refined abstract domain $\Symbolic \otimes \adom{D}$.  The
analysis of $\mathit{k} : \code{z} \takes \code{e2op}$ with an
abstract element $\tupla{\ael{s}, \ael{d}}$, first uses \ael{s} to
refine \code{e2op} to an expression $\code{e2op}^+$, then it performs
the assignment over the basic numerical domain: $\adom{D}.\aassign(\ael{d}, \code{z}, \code{e2op}^+)$.


\subsection{Assumptions and Assertions}
We consider just the \Assume\ statement, the case for \Assert\ being similar.
At source level, the PSA just passes the expression to be assumed to the underlying domain:
\begin{equation*}
\HLSem{\Assume\ \code{e};} = \lambda \ael{d}. \adom{D}.\atest(\ael{d}, \code{e}).
\end{equation*}
The compilation generates code to evaluate the condition \code{e}, and it assumes the result:
\begin{equation}
  \Comp(\Assume \code{e};) = \left[ \begin{array}{ll}&\CompExp(\code{e})\\ \mathit{k} :&   \Assume\ \result  \end{array} \right. \ .
\label{eq:compileassume}
\end{equation}
The bytecode semantics passes the literal to the underlying abstract domain:
\begin{equation*}
\LLSem{\mathit{k} : \Assume\ \code{l}} = \lambda \ael{d} \in \adom{D}.\atest(\ael{d}, \code{l}).
\end{equation*}
The compilation schema (\ref{eq:compileassume}), which is common to \eg, the C\# and Java compilers, introduces severe imprecision in analyses, as illustrated by Ex.~\ref{ex:intro} and by:

\begin{example}[Precision Loss in Tests]
Consider the statement $\mathcal{D} \equiv \Assume\ 0 \leq \code{x};$
to be analyzed with \Octagons, in the initial state $\atp_\Octagons =
\emptyset$.  Then, $\HLSem{ \Assume\ 0 \leq \code{x};}(\atp_\Octagons)
= \{ -\code{x} \leq 0 \}$.  The compilation of $\mathcal{D}$ is
\begin{equation}
  \Comp( \Assume\ 0 \leq \code{x};) = \left [
    \begin{array}{r l}
      0: & \result \takes 0 \leq \code{x} \\
      1: & \Assume \result
    \end{array} \right. .
  \label{ex:comp3}
\end{equation}
At program point $0$, \result\ is assigned the result of
evaluating the boolean condition. Since nothing is known in the
input state about \code{x}, nothing can be concluded about the
truth of \code{0 \leq \code{x}}, and hence \result \ is unconstrained.
As a consequence, $\LLSem{\Comp(\Assume\ 0 \leq
\code{x};)}(\atp_\Octagons) = \atp_\Octagons$. \qed
\end{example}
The previous example shows that strong relative completeness does not
hold.  If we analyze (\ref{ex:comp3}) with \Polyhedra, the situation
does not change, because even \Polyhedra\ cannot capture the relation
between a variable and the truth value of an expression.  Thus, if we
seek weak relative completeness, we need to refine the abstract
domain with either an abstract domain for tracking boolean
expressions, or more generally use the symbolic abstract domain
\Symbolic\ introduced in Sect~\ref{sec:assignment} to
``reconstruct'' larger expressions, that can then be passed to
the underlying numerical abstract domain.

Whereas in Sect~\ref{sec:assignment} the use of \Symbolic\ was just an
alternative w.r.t. the use of a more precise numerical domain, it
becomes a \emph{necessity} for handling boolean expressions.  
The use of the symbolic domain during low-level analysis requires a refinement of the
transfer functions, as shown by the next example.

\begin{example}[Precision Loss Induced by Compilation]
Consider a slight modification of the previous example:  $\mathcal{F} \equiv \Assume\ !(0 \leq \code{x});$ to be analyzed with \Octagons, in the entry state $\atp_\Octagons$.
$\HLSem{ \Assume\ !(0 \leq \code{x});}(\atp_\Octagons) = \{ \code{x} \leq -1 \}$.
The compilation of $\mathcal{F}$ (\eg, by C\#) is 
\begin{equation}
  \Comp( \Assume\ !(0 \leq \code{x});) = \left [
    \begin{array}{r l}
      0: & \code{t} \takes 0 \leq \code{x} \\
      1: & \result \takes \code{t} == 0 \\
      2: & \Assume \result
    \end{array} \right. .
  \label{ex:comp4}
\end{equation}
At program point $2$, the analysis of the compiled code, using the refined domain $\Symbolic
\otimes \Octagons$ infers the abstract
state $\ael{r} = \tupla{[\code{t} \mapsto 0 \leq \code{x}, \result
\mapsto \code{t} == 0], \atp_\Octagons} $.  Then, $\result$ is refined
to the expression $\result^+ \equiv (0 \leq \code{x})== 0$, which
cannot be generated by the syntax in Fig.~\ref{fig:WhileLanguage}.  As
a consequence, $\Octagons.\aassign$, designed for the high level, does
not understand $\result^+$, and hence ignores it:
$\LLSem{\Comp(\Assume\ !(0 \leq \code{x});)}(\tupla{\atp_\Symbolic,
\atp_\Octagons}) = \ael{r}$.  \qed
\end{example}

\subsubsection{Discussion: Refining the Transfer Functions, and Program Transformations.}
The example above underlines the fact that, in order to obtain weak
completeness, one must also refine the transfer functions.  For
instance, in the example $\Octagons.\aassign$ must be refined to
perform the semantic preserving rewritings $(0 \leq \code{x})== 0
\rightsquigarrow !(0 \leq \code{x}) \rightsquigarrow \code{x} < 0$.

In practice, a PSA designer has two choices: perform the rewriting
phase online or offline.  In the first case, a transfer function first
rewrites the boolean expressions, \eg, by applying the De Morgan laws,
by rewriting $\code{e} == 0$ as $!(\code{e})$, etc., and then
proceeds.  In the second case, in a pre-processing step, a program
$\mathcal{S}$ is analyzed with just \Symbolic, all the expressions in
$\mathcal{S}$ are first refined and then simplified as above, to
obtain a refined program $\mathcal{S}^+$.  Then, $\mathcal{S}^+$
is analyzed using \adom{D}.
In \Clousot{}, we have adopted the first approach.

\subsection{Conditionals}
The analysis of conditional statements (i) refines the input abstract
state with the guard, (ii) analyzes the two branches in the refined
state, and (iii) joins the results at the exit point.  Precise
handling of guards is essential for a PSA.
\begin{equation}
  \begin{array}{l}
    \HLSem{ \If(\code{e})~ \{ \Stm_1 \} \Else~\{ \Stm_2 \};} =  \\
    \quad \quad \quad \lambda \ael{d}. \HLSem{\Stm_1}(\adom{D}.\atest(\ael{d}, \code{e}))\ \acup\ \HLSem{\Stm_2}(\adom{D}.\atest(\ael{d}, \code{!(e)})).
  \end{array}
  \label{hl:if}
\end{equation}
One possible compilation is:
\begin{equation}
  \Comp( \If(\code{e})~ \{ \Stm_1 \} \Else~\{ \Stm_2 \}; ) = 
  \left [ 
    \begin{array}{r l}
      & \Compexp(\code{e}) \\
      k :& \code{b} \takes \result == 0 \\
      k+1 :& \JumpIfTrue\ \code{b}\  t \\
      & \Comp(\Stm_1) \\
      & \Jump\ \mathit{out} \\
      t: & \Comp(\Stm_2)\\
      \mathit{out}: &\Nop
    \end{array}
  \right. \ .
\label{comp:If}
\end{equation}
The low level analysis of (\ref{comp:If}) can be made very similar to
(\ref{hl:if}), provided that some preprocessing of the bytecode is
performed.  The first step is to construct the control flow graph from
(\ref{comp:If}), as in Fig.~\ref{fig:cfgIf}.  However, that is not
enough, because one wants to know that $!(\code{b})$
(resp. $\code{b}$) holds at program point $k+2$ (resp. $t$).
Propagating such an information during a dataflow analysis is
non-trivial.

\begin{figure}[t]
  \[
  \xymatrix @-1.2pc {
    & \fbox{$\begin{array}{rl}&\Compexp(\code{e}) \\  k:& \code{b} \takes \result == 0 \\ k+1 :& \JumpIfTrue\ \code{b}\  t  \end{array}$} \ar[dr] \ar[dl] \\
    \fbox{$\Comp(\Stm_1)$} \ar[dr]&  &\fbox{$\Comp(\Stm_2)$}\ar[dl]  \\
    &\fbox{$\mathit{out}: \Nop$} &
  }
  \]
  \caption{The control flow graph constructed from $\Comp( \If(\code{e})~ \{ \Stm_1 \} \Else~\{ \Stm_2 \}; )$.}
  \label{fig:cfgIf}
\end{figure}

\comment{
\begin{figure}[t]
  \[
  \xymatrix @-1.2pc  {
    &\fbox{$\begin{array}{rl}&\Compexp(\code{e}) \\  k:& \code{b} \takes \result == 0 \end{array}$} \ar[dr] \ar[dl] \\
    \fbox{$\begin{array}{rl} &\Assume \code{b} \\ &\Comp(\Stm_1) \end{array}$}\ar[dr]&  &\fbox{$\begin{array}{rl} &\Assume \code{!(b)} \\ &\Comp(\Stm_2) \end{array}$} \ar[dl]  \\
    &\fbox{$\mathit{out}: \Nop$} &
  }
  \]
  \caption{The enhanced control flow graph constructed from $\Comp( \If(\code{e})~\allowbreak \{ \Stm_1 \} \allowbreak \Else~\{ \allowbreak \Stm_2\allowbreak \}; )$, in which the assumptions at the head of the conditional branches are made explicits.}
  \label{fig:cfgIf2}
\end{figure}
}

A better approach is to provide another view of the code
(\ref{comp:If}), in which the guard of the conditional is made
explicit in the true-branch and the false-branch as \code{assume}
statements.  This is the direction
we have taken in \Clousot{}.  In general, let $\fbox{\code{B}}$
the block which computes the truth value of the guard \code{e},
$\fbox{\code{T(e)}}$ and $\fbox{\code{F(e)}}$ the (compilation of the)
two branches of the conditional dominated by (resp.) $\Assume
\code{b}$ and $\Assume \code{!(b)}$, and $\fbox{\code{O}}$ be the exit
block.  Then the low level semantics can be defined as:
\begin{equation}
\def\objectstyle{\scriptstyle}
\def\labelstyle{\scriptstyle}
\begin{array}{l c l}
  \BigLLSem{\raisebox{0.6cm}
    {
      \xymatrix @-1.8pc  {
        &\fbox{\code{B}} \ar[dl] \ar[dr]& \\
        \fbox{\code{T(e)}} \ar[dr]& & \fbox{\code{F(e)}}\ar[dl] \\
        & \fbox{\code{O}} &
      }
    }
  } & = & 
  \begin{array}{l}
    \lambda \ael{r} \in \adom{D} \otimes \Symbolic.\ \\
    \quad \quad \quad \mathrm{let}\ \ael{r}_1 = \LLSem{\code{B}}(\ael{r})\ \mathrm{in} \\
    \quad \quad \quad \mathrm{let}\ \ael{r}_t = \LLSem{\code{T(e)}}((\adom{D} \otimes \Symbolic).\atest(\ael{r}_1, \code{e}))\ \mathrm{in} \\
    \quad \quad \quad \mathrm{let}\ \ael{r}_f =  \LLSem{\code{F(e)}}((\adom{D} \otimes \Symbolic).\atest(\ael{r}_1, \code{!(e)}))\ \mathrm{in}\\
    \quad \quad \quad \mathrm{in}\ \ael{r}_t \acup \ael{r}_f \ . 
  \end{array}
\end{array}
\end{equation}

However, incompleteness can still show up if the compilation scheme is different from  (\ref{comp:If}), in particular for the handling of expressions.
The next example is inspired by the way the C\# compiler \cite{VisualCSharp}, generates code for shortcutting boolean expressions.

\begin{example}[Loss of Precision Induced by Compilation of Shortcut Expressions]
Let $\mathcal{G}$ be the code snippet $\If (0 \leq \code{i}\ \&\&\ \code{i} < \code{len})\ \{ \Stm_1 \}\ \Else\ \{ \Stm_2 \}$.
The $C\# 2.0$ compiler generates code that looks like the one in Fig.~\ref{comp:shortcut}.
Briefly, if one of  the operands of $\&\&$ is false, then it jumps to line $8$, which sets  $\result$ to $0$ .
Otherwise, it sets $\result$ to $1$.
The two flows are then merged at program point $9$, which implies that $\result == 0$ and $\result == 1$ are joined, \ie, the information about the truth of the guard, $\result == 0 \Longleftrightarrow !(0 \leq \code{i}\ \&\&\ \code{i} < \code{len})$ and $\result == 1 \Longleftrightarrow (0 \leq \code{i}\ \&\&\ \code{i} < \code{len})$ is lost.
So it cannot be further propagated in the two branches of the conditional. \qed
\end{example}
The incompleteness in the previous example can be resolved either by precisely modeling the relation between boolean variables and boolean expressions with BDDs as in \cite{Jeannet02}, or by approximating the double implication with a simple implication, \eg, using trace partitioning, \cite{Tzolovski98}.
As a consequence, the underlying abstract domain must be refined to the reduced cardinal power $\parti{\Lit} \rightarrow (\adom{D} \otimes \Symbolic)$, so as to obtain the weak relative completeness for shortcut conditionals.

\begin{figure}[t]
\[
\begin{array}{ccc}
\multicolumn{3}{c}{
 \Comp(\If (0 \leq \code{i}\ \&\&\ \code{i} < \code{len})\ \{ \Stm_1
 \} \Else \{ \Stm_2 \}) = }
\\[5mm]
%\left [
  \begin{array}[b]{r l}
    0: & \code{t}_1 \takes 0 \leq \code{i} \\
    1: & \code{b}_1 \takes \code{t}_1 == 0 \\
    2: & \JumpIf\ \code{b}_1\ 8 \\
    3: & \code{t}_2 \takes \code{i} < \code{len} \\
    4: & \code{b}_2 \takes \code{t}_2 == 0 \\
  \end{array}
%\right.
&
  \hspace{5mm}
  \begin{array}[b]{r l}
    5: & \JumpIf\ \code{b}_2\ 8 \\
    6: & \result\ \takes 1 \\
    7: & \Jump\ 9 \\
    8: & \result\ \takes\ 0 \\
  \end{array} 
&
  \hspace{5mm}
  \begin{array}[b]{r l}
    9: & \JumpIf\ \result\ \mathit{k}+1 \\
    10: & \Comp(\Stm_2) \\
    \mathit{k} : & \Jump\ \mathit{out} \\
    \mathit{k}+1 : & \Comp(\Stm_1) \\
    \mathit{out} : &\Nop
  \end{array}
\end{array}
\]
\caption{The (simplified version of the) code generated by the $C\# 2.0$ compiler for the statement  $\If (0 \leq \code{i}\ \&\&\ \code{i} < \code{len})\ \{ \Stm_1 \} \Else \{ \Stm_2 \}$.}
\label{comp:shortcut}
\end{figure}

\subsection{Loops}
The semantics of a loop is given as a least fixpoint over a suitable partial order:
\begin{equation*}
  \begin{array}{r l}
    \HLSem{\While(\code{e})\ \{\ \Stm\ \};} = \lambda \ael{d}.\ & \mathrm{let}\ \ael{inv} = \mathsf{lfp}_{\abot}^{\aless} \lambda X.\ \ael{d} \acup \HLSem{\Stm}(\adom{D}.\atest(X, \code{e})) \\
& \mathrm{in}\ \adom{D}.\atest(\ael{inv}, \code{!(e)}).
  \end{array}
\end{equation*}
The least fixpoint equals the limit of the increasing iterations starting from $\abot$.
In general the iterations may not converge, so that a widening operator~\cite{CousotCousot77} is used to force convergence to a post-fixpoint.
Then, a narrowing operator~\cite{CousotCousot77} is applied to recover some precision.
An easy yet generic and useful form of narrowing is given by doing one more iteration starting from the post-fixpoint, as shown by the next example.

\begin{example}[Narrowing by Re-Execution]
\label{ex:narrowing}
Let $\mathcal{W} \equiv \code{z} := 0;\ \While(\code{z} < 100 )\ \{\ \code{z} := \code{z} + 1; \}; \Assert \code{z} == 100;$ and let us analyze it with the \Intervals\ abstract domain.
The fixpoint iterations produce the increasing chain of intervals $[0,0] \aless\allowbreak [0,1] \aless\allowbreak [0, 2]\allowbreak \dots \aless [0, n]$, which is extrapolated by the standard widening on intervals to $[0, +\infty]$, so that $\el{inv}^\widening = [\code{z} \mapsto [0,  +\infty]] $ is an invariant for the loop.
On the other hand, it is not precise enough to prove the assertion after the loop.
By \emph{first} re-executing the body starting from the fixpoint, one gets $[0,0] \acup [1, 100] = [0, 100]$, so that  $\el{inv}^\narrowing = [\code{z} \mapsto [0,  100]]$.
\emph{Then},  $\el{inv}^\narrowing$ intersected with the negation of the loop guard is enough to prove the assertion. \qed
\end{example}

The compilation of a while statement looks like
\begin{equation}
  \Comp(\While(\code{e})\ \{\ \Stm\ \}; ) = 
  \left [ 
    \begin{array}{r l}
      b: & \Compexp(\code{e}) \\
      k :& \code{b} \takes \result == 0 \\
      & \JumpIfTrue\ \code{b}\  \mathit{out} \\
      & \Comp(\Stm) \\
      & \Jump\ b \\
      \mathit{out}: &\Nop
    \end{array}
  \right. \ .
\label{comp:While}
\end{equation}

A typical analysis of the unstructured code above first detects the
back edges, in order to find the program points where widening is needed.
However, back edges detection is not enough to ensure relative completeness when extrapolating operators are used, as shown by the next example.

\begin{example}[Narrowing by Re-Execution, continued]
The CFG graph for $\mathcal{W}$ is in Fig.~\ref{fig:narrowing}.  A
standard back-edges analysis detects that the block starting at $1$ is
the target of a back edge, and hence the widening point.  Then, we
analyze the program on the domain $\Intervals \otimes \Symbolic$, and
we infer the invariant $\code{z} \mapsto [0, +\infty]$ at program
point $1$.  Now we want to refine it using the re-execution based
narrowing.  In the source level case, we just proceeded by induction
on the structure.  At the low-level, we don't know which edge leads
into the loop, and which edge leads out of the loop. If we push the
invariant \emph{first} onto the left branch (\ie, on program point
$4$), then we obtain the desired refined
$\code{z} \mapsto [0, 100]$, which is then pushed onto the right
branch, where it is enough to prove the assertion is not violated.
On the other hand, if we push the invariant \emph{first} onto the right
branch (\ie, on program point $6$), we obtain no invariant refinement.
\qed
\end{example}

\comment{
\subsubsection{Discussion: Choosing the right narrowing operator.}
From the example above, it follows that in the low level world the use
of the re-execution based narrowing must be done with some care.  The
example implies that \emph{one} (decreasing) iteration is not enough
to ensure weak relative completeness.  An immediate solution to this
problem is to iterate the decreasing iterations until stabilization of
the full CFG.  The problem with such an approach is that it may
introduce unuseful computations.  For instance, in the example the
``right'' branch may be arbitrarily complex, so that the analysis may
go deep along that branch, and then because of the refinement induced
by the ``left'' branch, the same branch must be re-analyzed.  In the
worst case, this may introduce a quadratic (w.r.t. the number of
branches in the graph) slowdown of the analysis.  An improvement is to
pre-process the CFG, so to annotate each outgoing edge with
information whether there is a loop starting from that edge, and use
this information to schedule the blocks during the narrowing phase.

The preprocessing can be combined with domain-specific narrowings, so
to improve the overall precision of the analysis.  } The example shows
that applying standard narrowing techniques from source level analysis
is tricky on low-level code, as the necessary high-level loop
structures are not apparent. Symbolic expression recovery is not
sufficient, as control flow is involved. Thus, to obtain relative
completeness for loops, some form of loop recovery must be performed.

\begin{figure}[t]
  \[
  \xymatrix @-1.2pc {
    & \fbox{$\begin{array}{rl} 0: \code{z} := 0\end{array}$} \ar[d] \\
    & \fbox{$\begin{array}{rl} 
        1: &  \Nop \\
        2 :& \result \takes \code{z} < 100 \\
        3: & \code{b} \takes \result == 0 \\
      \end{array}$} \ar[dr] \ar@/^/@<1ex> [dl] \\     
    \fbox{$\begin{array}{rl} 
        4 : & \Assume\ \code{!(b)} \\  
        5 : & \code{z} := \code{z} + 1 \end{array} $} \ar@/^/@<1ex> [ur]
    &  &\fbox{$\begin{array}{rl} 6 : & \Assume\ \code{b}\\ 7 : & \Assert \code{z} == 100 \end{array}$}  \\   
  }
  \]
\caption{The enhanced CFG graph for the three addresses compilation for the code in Ex.~\ref{ex:narrowing}.
Exact narrowing requires the knowledge that the left branch leads to a cycle.}
\label{fig:narrowing}
\end{figure}

\section{Conclusions}
We have presented a series of issues faced by low-level code analyzers
if their precision is to match the precision typically achieved by a
source analysis.
We have formalized the relation between the low-level and high-level
analyses via the concepts of strong and weak relative completeness.  
By analysis on the program constructs, we have shown: (i) how strong
relative completeness can be obtained only for trivial cases, and (ii)
how weak relative completeness can be obtained by refining
the underlying domain for the analysis, the transfer functions, and
by pre-processing of the program.  However, it turns out that
the refinement step must be handled with care by the designer of the
precise static analysis, in order to avoid transforming a polynomial
problem (\eg, the analysis of the source program with Octagons) into
an exponential one.

\bibliographystyle{plain}
\small
\bibliography{bib}



\end{document}
